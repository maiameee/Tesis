\documentclass[11pt,a4paper]{article}
\usepackage[utf8]{inputenc}
\usepackage{graphicx}
\usepackage[spanish]{babel} 
\usepackage{textcomp}
\usepackage{float}
\usepackage{subfig}
\usepackage{multicol}
\usepackage[left=2cm,top=2.5cm,right=2cm,bottom=2.5cm]{geometry} 
\usepackage{lipsum}
\usepackage{circuitikz}
\usepackage{fancyhdr}
\usepackage{amsmath}
\begin{document}

\title{\Huge Resumen de Anisotropic fluid dynamics for gubser flow}
\author{\huge  arXiv:1703.10955 [nucl-th]}





\date{Abril 2017}

\renewcommand{\headrulewidth}{0.5pt}




\maketitle

La idea es generalizar Bjorken a un flujo con parte transversa finita y con expansión transversal y en el eje del haz

\section{Introducción a flujo de Gubser}

\subsection{Flujo de Bjorken}
El esquema de Bjorken asume invariancia frente a boosts a lo largo del eje de haz, invariancia de traslación y rotación en el plano transverso (la invariancia de traslación es en la coordenada radial transversal). Con estas simetrías podemos parametrizar Minkowski como:

\begin{equation}
ds^2=-d{\tau}^2+{\tau}^2d{\eta}^2+dx^{2}_{\bot}+x^{2}_{\bot}d{\phi}^2
\end{equation}

Las invariancias frente a traslaciones y rotaciones implican que nada puede depender de $x_{\bot}$ o ${\phi}$, mientras que la invariancia frente a boost (si es exacta) implica que nada depende de $\eta$. Junto con la invariancia ante reflexiones en el eje del haz ($\eta\Rightarrow-\eta$) estas simetrías implican que en las coordenadas $(\tau,\eta,x_{\bot},\phi)$ el vector cuadrivelocidad queda completamente determinado, siendo $u^{\mu}=(1,0,0,0)$. Debido a las simetrías del problema no necesitamos una ecuación de estado para determinar el perfil de $u^{\mu}$.

La invariancia ante traslaciones en el plano transverso no es algo realista, ya que el radio de los núcleos en las colisiones es de aproximadamente $13 fm$. La idea de Gubser es hacer que la dinámica de la colisión y el estado inicial respeten perfectamente la invariancia conforme relativista. La simetría conforme es debido a que los procesos en QCD por encima de la escala de confinamiento son aproximadamente invariantes conforme. Para extraer el perfil de velocidades reemplaza la simetría frente a traslaciones en $x_{\bot}$ por una simetría ante una especie de transformaciones conformes combinadas con rotaciones $SO(3)$ alrededor del eje del haz.

\subsection{$SO(3)$ como subgrupo del grupo conforme}

El grupo conforme es una extensión del grupo de Poincaré $ISO(3,1)$ a $SO(4,2)$ y los generadores del álgebra son:

\begin{itemize}
\item  Traslaciones: $\xi^{\mu}=a^{\mu}$ con $a^{\mu}$ constante.
\end{itemize}

\begin{itemize}
\item  Rotaciones: $\xi^{\mu}=\omega^{\mu}_{\nu}x^{\nu}$ con $\omega_{{\mu}{\nu}}$ constante y antisimétrico.
\end{itemize}

\begin{itemize}
\item  Transformaciones de escala: $\xi^{\mu}=x^{\mu}$.
\end{itemize}

\begin{itemize}
\item  Transformaciones conformes especiales: $\xi^{\mu}=x^{\nu}x_{\nu}b^{\mu}-2b^{\nu}x_{\nu}x^{\mu}$ con $b^{\mu}$ constante.
\end{itemize}


Las primeras dos pertenecen a Poincaré y son vectores de killing $\textbf{\textit{L}}_{\xi}g_{{\mu}{\nu}}=0$. Los campos asociados a las transformaciones conformes y de escala no son vectores de killing pero son vectores conformes de killing y cumplen $\textbf{\textit{L}}_{\xi}g_{{\mu}{\nu}}=\frac{1}{2}(\nabla_{\lambda}\xi^{\lambda})g_{{\mu}{\nu}}$ (generan transformaciones que dejan invariante la métrica salvo transformaciones conformes).

\subsection{Perfil de $u^{\mu}$}


\section{El flujo de Gubser}

El flujo de Gubser es mas fácil de describir en el espacio de de Sitter de dimensión tres mas una linea $ds_{3}{\otimes}R$ donde el flujo se ve estático. Para pasar de Minkowski en coordenadas de Milne a $ds_{3}{\otimes}R$ hacemos un reescalado de Weyl de la métrica, seguido de un cambio de coordenadas $x^{\mu}=(\tau,\eta,x_{\bot},\phi)\rightarrow{\hat{x}^{\mu}}=(x^{0},x^{1},x^{2},x^{3})=(\rho,\theta,\phi,\eta)$ de la forma:


\begin{subequations}
\begin{align}
\rho(\tilde{\tau},\tilde{r})=-\arcsin(\frac{1-{\tilde{\tau}}^2+{\tilde{r}}^2}{2\tilde{\tau}})\\
\theta(\tilde{\tau},\tilde{r})=\arctan(\frac{2\tilde{r}}{1+{\tilde{\tau}}^2-{\tilde{r}}^2})
\end{align}
\label{cambio_de_coordenadas}
\end{subequations}

con $\tilde{\tau}=q\tau$ y $\tilde{r}=qr$, donde q es una escala de energía arbitraria que fija el tamaño transversal del sistema. En estas coordenadas en elemento de linea queda



\begin{equation}
d\hat{s}^2=-d{\rho}^2+\cosh^2{\rho}(d{\theta}^2+\sin^2{\theta}d{\phi}^2)+d{\eta}^2
\label{metrica_de_Sitter}
\end{equation}

Este elemento de linea (\ref{metrica_de_Sitter}) es invariante ante rotaciones en el espacio $(\theta,\phi)$, que se corresponde con el grupo de simetría $SO(3)_q$. Agregando la simetría de reflexión, el elemento de linea (\ref{metrica_de_Sitter}) es invariante ante la "simetría de Gubser" $SO(3)_{q}{\otimes}SO(1,1){\otimes}Z_{2}$. El único vector normalizado y time-like en esta simetría es $u^{\mu}=(1,0,0,0)$.
La simetría de Gubser también implica que las variables macroscopicas como la densidad de energía $\hat{\epsilon}(\hat{x})=\hat{\epsilon}(\rho)$ sólo depende del tiempo de de Sitter $\rho$. Además la función distribución $f( \hat{x}, \hat{p})=f( \rho, \hat{p}^2_{\Omega},\hat{p}^2_{\eta})$ solo puede depender del tiempo de de Sitter y de las componentes de momento $\hat{p}^2_{\Omega}=\hat{p}^2_{\theta}+\hat{p}^2_{\phi}/\sin^2\theta$.

\section{Dinámica del flujo de Gubser}




\section{Hidrodinámica viscosa}

Para obtener la hidrodinámica viscosa estándar expanden la función distribución cinética alrededor del equilibrio local.

\begin{equation}
f( \hat{x}, \hat{p})= f_{eq}(\beta_{\hat{u}}(\hat{x})(-\hat{u}(\hat{x}).\hat{p}))+{\delta}f( \hat{x}, \hat{p})
\label{expansion}
\end{equation}

Con $\beta_{\hat{u}}=\frac{1}{\hat{T}(\hat{x})}$ y en el sistema de referencia local (LRF) $-\hat{u}(\hat{x}).\hat{p}$ es isótropo en el espacio de los momentos. Acá ${\delta}f$ es el que llevas la información las desviaciones del equilibrio local, en particular de las desviaciones del las anisotropías locales causadas por una expansión global anisótropa.



En el marco de Landau la forma mas general del tensor de energía-momento es:

\begin{equation}
\hat{T}^{{\mu}{\nu}}= \hat{\epsilon}\hat{u}^{\mu}\hat{u}^{\nu}+\hat{P}\hat{\Delta}^{{\mu}{\nu}}+\hat{\pi}^{{\mu}{\nu}}
\end{equation}

Con $\hat{\epsilon}$ la densidad de energía en LRF, $\hat{P}=\hat{P}_{0}+\hat{\Pi}$ es la presión isótropa y $\hat{\pi}^{{\mu}{\nu}}$ es el tensor de esfuerzos viscosos.

Para sistemas con simetría conforme la presión de bulk $\hat{\Pi}$ es nula. Esto se debe a que la presión de bulk es una fuerza viscosa que se opone a la expansión o compresión del fluido, lo cual se contradice con la simetría conforme (o al revés). En estos casos la presión isótropa  $\hat{P}$ es la presión térmica $\hat{P}_{0}=\hat{\epsilon}/3$.

Las cantidades macroscópicas que aparecen en el tensor de energía-momento son proyecciones de los momentos de la función distribución $f( \hat{x}, \hat{p})$

\begin{equation}
\hat{\epsilon}=\hat{u}_{\mu}\hat{u}_{\nu}\hat{T}^{{\mu}{\nu}}=<(\hat{u}(\hat{x}).\hat{p})^2>
\end{equation}
\begin{equation}
\hat{P}=\frac{1}{3}\hat{\Delta}_{{\mu}{\nu}}\hat{T}^{{\mu}{\nu}}= \frac{1}{3}\hat{\Delta}_{{\mu}{\nu}}\hat{p}^{\mu}\hat{p}^{\nu}
\end{equation}
\begin{equation}
\hat{\pi}^{{\mu}{\nu}}= \hat{T}^{<{\mu}{\nu}>}=<\hat{p}^{<\mu}\hat{p}^{\nu>}>
\end{equation}

Donde $-\hat{u}_{\mu}\hat{u}_{\nu}$ es un proyector en la dirección de la cuadrivelocidad y $\hat{\Delta}_{{\mu}{\nu}}=\hat{g}_{{\mu}{\nu}}+\hat{u}_{\mu}\hat{u}_{\nu}$  proyecta en la superficie ortogonal a $\hat{u}_{\mu}$. Debido a que en el LRF tenemos $\hat{u}_{\mu}=(1,0,0,0)$, estos proyectores se llaman locamente temporal y espacial respectivamente.
Por último el tensor de esfuerzos viscosos $\hat{\pi}^{{\mu}{\nu}}$ es la proyección de $\hat{T}^{{\mu}{\nu}}$ ortogonal a $\hat{u}_{\mu}$ y sin traza. Este proyector tal que $\hat{\Delta}^{{\mu}{\nu}}_{{\alpha}{\beta}}=\frac{1}{2}(\hat{\Delta}^{\mu}_{\alpha}\hat{\Delta}^{\nu}_{\beta}+\hat{\Delta}^{\mu}_{\beta}\hat{\Delta}^{\nu}_{\alpha}-\frac{2}{3}\hat{\Delta}^{{\mu}{\nu}}\hat{\Delta}_{{\alpha}{\beta}})$ y $\hat{B}^{<{\mu}{\nu}>}=\hat{\Delta}^{{\mu}{\nu}}_{{\alpha}{\beta}}\hat{B}^{{\mu}{\nu}}$.

La unicidad de la descomposición (\ref{expansion}) requiere fijar la inversa de la temperatura local $\beta_{\hat{u}}(\hat{x})$, lo cual se realiza gracias al la condición de Landau:

\begin{equation}
\hat{\epsilon}=<(\hat{u}(\hat{x}).\hat{p})^2>_{eq}=\hat{\epsilon}_{eq}({\hat{T}})=\frac{3}{\pi^2}{\hat{T}}^4
\end{equation}

De esta forma ${\hat{T}}$ en $f_{eq}$ es ajustada de forma que ${\delta}f$ no contribuya  a la densidad de energía en LRF $(\hat{\epsilon})$. Debido a esto toda la información de las desviaciones del sistema del equilibrio local esta guardada en el tensor de esfuerzos viscosos $\hat{\pi}^{{\mu}{\nu}}=<\hat{p}^{<\mu}\hat{p}^{\nu>}>_{{\delta}f}$ (es un momento de ${{\delta}f}$)


\subsection{Ecuación para la densidad de energía}
La ecuación de evolución para la densidad de energía se obtiene proyectando sobre $u_{nu}$ (proyección temporal) la ecuación de conservación del tensor energía-momento ($\hat{D}_{\mu}\hat{T}^{{\mu}{\nu}}=0$), donde $\hat{D}_{\mu}$ es la derivada covariante.
Usando la conexión de Levi-Civita, los símbolos de Christoffel la métrica (\ref{metrica_de_Sitter}) son:

\begin{subequations}
\begin{align}
\label{011}
\Gamma^{0}_{{1}{1}}=\cosh(\rho)\sinh(\rho)\\
\label{022}
\Gamma^{0}_{{2}{2}}=\cosh(\rho)\sinh(\rho)\sin^2(\theta)\\
\Gamma^{1}_{{2}{2}}=\cos(\theta)\sin(\theta)\\
\Gamma^{1}_{{1}{0}}=\Gamma^{1}_{{0}{1}}=\Gamma^{1}_{{2}{0}}=\Gamma^{1}_{{0}{2}}=\tanh(\rho)\\
\Gamma^{2}_{{2}{1}}=\Gamma^{2}_{{1}{2}}=-\cot(\theta)
\end{align}
\label{christoffel}
\end{subequations}

Utilizando que $u_{\nu}=(-1,0,0,0)$ la proyección queda:

\begin{equation}
u_{\nu}\hat{D}_{\mu}\hat{T}^{{\mu}{\nu}}= \hat{T}^{{\mu}{0}}_{,\mu} +\Gamma^{\mu}_{{j}{\mu}}\hat{T}^{{j}{0}}+\Gamma^{0}_{{j}{\mu}}\hat{T}^{{\mu}{j}}
\label{cons_proy}
\end{equation}

Dado que $\hat{T}^{{\mu}{0}}=u_{\nu}\hat{T}^{{\mu}{\nu}}=\hat{\epsilon}\hat{u}^{\mu}$ el primer termino queda $\hat{T}^{{\mu}{0}}_{,\mu}={\partial}_{\rho}\hat{\epsilon}$. El segundo termino solo aporta si $j=0$, por lo tanto $\Gamma^{\mu}_{{0}{\mu}}\hat{T}^{{0}{0}}=\Gamma^{1}_{{0}{1}}\hat{\epsilon}+\Gamma^{2}_{{0}{2}}\hat{\epsilon}=2\tanh(\rho)\hat{\epsilon}$.
Finalmente desarrollamos el tercer termino:

\begin{equation}
\Gamma^{0}_{{j}{\mu}}\hat{T}^{{\mu}{j}}=\hat{\epsilon}\Gamma^{0}_{{j}{\mu}}\hat{u}^{j}\hat{u}^{\nu}+\hat{P}\Gamma^{0}_{{j}{\mu}}\hat{\Delta}^{{\mu}{j}}+\Gamma^{0}_{{j}{\mu}}\hat{\pi}^{{\mu}{j}}
\label{tercterm}
\end{equation}

Se puede ver que el primer termino de (\ref{tercterm}) se anula ya que $\Gamma^{0}_{{j}{\mu}}\hat{u}^{j}\hat{u}^{\nu}=\Gamma^{0}_{{0}{0}}=0$. Ademas al desarrollar en el segundo termino el proyecotr $\hat{\Delta}^{{\mu}{j}}$, hay una parte que se anula ($\hat{P}\hat{u}^{j}\hat{u}^{\nu}$). El segundo término termina dando $\hat{P}\Gamma^{0}_{{j}{\mu}}\hat{\Delta}^{{\mu}{j}}=\hat{P}(\Gamma^{0}_{{1}{1}}\hat{g}^{{1}{1}}+\Gamma^{0}_{{2}{2}}\hat{g}^{{2}{2}})=2\tanh(\rho)\hat{P}_{0}=\frac{2}{3}2\tanh(\rho)\hat{\epsilon}$, donde en el ultimo paso se uso que $\hat{P}=\hat{P}_{0}=\epsilon/3$ cumple la ecuación de estado conforme.\\





En el último término de (\ref{tercterm}) desarrollamos la suma sobre los símbolos de Christoffel no nulos:

\begin{equation}
\Gamma^{0}_{{j}{\mu}}\hat{\pi}^{{\mu}{j}}=\Gamma^{0}_{{1}{1}}\hat{\pi}^{{1}{1}}+\Gamma^{0}_{{2}{2}}\hat{\pi}^{{2}{2}}=\cosh\rho\sinh\rho\hat{\pi}^{{1}{1}}+\cosh\rho\sinh\rho\sin^2\theta\hat{\pi}^{{2}{2}}
\label{shear_christoffel}
\end{equation}


Dado que el tensor de esfuerzos viscosos tiene que ser de traza nula (es la proyección espacial y de traza nula de $\hat{T}^{{\mu}{j}}$) esto nos introduce una nueva ecuación que permite simplificar la expresión de (\ref{shear_christoffel}):

\begin{equation}
Tr(\hat{\pi}^{{\mu}{\nu}})=\hat{g}_{{\mu}{\nu}}\hat{\pi}^{{\mu}{\nu}}=\cosh^2\rho\hat{\pi}^{{1}{1}}+\cosh^2\rho\sinh^2\theta\hat{\pi}^{{2}{2}}+\hat{\pi}^{{\eta}{\eta}}=0
\label{traza_shear}
\end{equation}


Sacando factor común $\tanh\rho$ en (\ref{shear_christoffel}) y usando la condición de la traza queda:


\begin{equation}
\Gamma^{0}_{{j}{\mu}}\hat{\pi}^{{\mu}{j}}= \tanh\rho(\cosh^2\rho\hat{\pi}^{{1}{1}}+\cosh^2\rho\sinh^2\theta\hat{\pi}^{{2}{2}})=-\hat{\pi}^{{\eta}{\eta}}\tanh\rho
\end{equation}


Reemplazado todos los términos calculados la ecuación de evolución de la densidad de energía resulta ser:

\begin{equation}
\partial_{\rho}\hat{\epsilon}+\frac{8}{3}\hat{\epsilon}\tanh\rho=\hat{\pi}^{{\eta}{\eta}}\tanh\rho
\label{ecuacion_energia}
\end{equation}

\subsection{Ecuación exacta para el tensor de esfuerzos viscosos}

Debido a la alta simetría de este sistema, el tensor de esfuerzos viscosos tiene una única componente independiente $\hat{\pi}\equiv\hat{\pi}^{{\eta}{\eta}}$. Para obtener la ecuación de evolución para $\hat{\pi}$ a partir de la ecuación de Boltzmann en aproximación de tiempo de relajación (RTA):

\begin{equation}
\frac{\partial\delta{f}}{\partial\rho}=-\frac{\delta{f}}{\tau_{rel}(\rho)}-\frac{\partial{f}_{eq}}{\partial\rho}
\end{equation}

Con $\hat{\tau}_{rel}(\rho)=\frac{\tau_{rel}(\rho)}{\hat{T}}=\frac{C}{\hat{T}(\rho)}$ y $\delta{f}=f-f_{eq}$ es el apartamiento de la función distribución del equilibrio local. El tensor de esfuerzos viscosos se escribe como:

\begin{equation}
\hat{\pi}^{<{\mu}{\nu}>}=\frac{1}{(2\pi)^3}\int{\frac{\partial^3\hat{p}}{\sqrt[]{-g}\hat{p}^{\rho}}}\hat{\Delta}^{{\mu}{\nu}}_{{\alpha}{\beta}}\hat{p}^{\alpha}\hat{p}^{\beta}\delta{f}
\label{boltzmann_df}
\end{equation}

Derivando la expresión de $\hat{\pi}^{<{\mu}{\nu}>}$ obtenemos

\begin{equation}
\partial_{\rho}\hat{\pi}^{<{\mu}{\nu}>}=
\hat{\Delta}^{{\mu}{\nu}}_{{\alpha}{\beta}}\partial_{\rho}\hat{\pi}^{{\alpha}{\beta}}=
\int{\frac{\partial^3\hat{p}}{(2\pi)^3}}\hat{\Delta}^{{\mu}{\nu}}_{{\alpha}{\beta}}\hat{p}^{\alpha}\hat{p}^{\beta}(\frac{1}{\sqrt[]{-g}\hat{p}^{\rho}}\frac{\partial\delta{f}}{\partial\rho}+\delta{f}\frac{\partial}{\partial\rho}(\frac{1}{\sqrt[]{-g}\hat{p}^{\rho}}))
\label{derivada_shear}
\end{equation}

El primer término de la integral lo podemos reescribir utilizando a ecuación de Boltzmann (\ref{boltzmann_df}) con $f_{eq}=exp({\frac{-\hat{p}^{\rho}}{\hat{T}(\rho)}})$. El segundo término se puede escribir desarrollando la derivada y utilizando que $\sqrt[]{-g}=\cosh^2\rho\cos\theta$ quedando:

\begin{equation}
\frac{\partial}{\partial\rho}(\frac{1}{\sqrt[]{-g}\hat{p}^{\rho}})=\frac{1}{\hat{p}^{\rho}}\frac{\sinh\rho}{\cosh^3\rho\cos\theta}+\frac{1}{(\hat{p}^{\rho})^2}\frac{1}{\sqrt[]{-g}}\frac{\partial\hat{p}}{\partial\rho}=
\frac{1}{\sqrt[]{-g}\hat{p}^{\rho}}\tanh\rho+\frac{1}{\hat{p}^{\rho}\sqrt[]{-g}}\frac{1}{\hat{p}^\rho}\frac{\partial\hat{p}}{\partial\rho}
\label{derivada_segundo_termino_shear}
\end{equation}


Reescribiendo (\ref{derivada_shear}) con Boltzmann y la ecuación (\ref{derivada_segundo_termino_shear}) obtenemos:


\begin{equation}
\partial_{\rho}\hat{\pi}^{<{\mu}{\nu}>}=
\int{\frac{\partial^3\hat{p}}{(2\pi)^3}}\hat{\Delta}^{{\mu}{\nu}}_{{\alpha}{\beta}}\hat{p}^{\alpha}\hat{p}^{\beta}(\frac{1}{\sqrt[]{-g}\hat{p}^{\rho}}(-\frac{\delta{f}}{\hat{\tau}_{rel}(\rho)}+\frac{\delta{f}_{eq}}{\partial\rho})
+\delta{f}(\frac{1}{\sqrt[]{-g}\hat{p}^{\rho}}\tanh\rho+\frac{1}{\hat{p}^{\rho}\sqrt[]{-g}}\frac{1}{\hat{p}^\rho}\frac{\partial\hat{p}}{\partial\rho}))
\label{derivada_shear_desarrollada}
\end{equation}

En esta integral hay términos proporcionales al tensor de esfuerzos viscosos (en los que se integra $\delta{f}$). Los factores que acompañan a estos términos son funciones de $\rho$ (tiempo de de Sitter) y por lo tanto salen fuera de la integral. Los términos restantes los dejamos como la expresión integral.


\begin{eqnarray}
&\partial_{\rho}\hat{\pi}^{<{\mu}{\nu}>}=-\frac{\hat{\pi}^{{\mu}{\nu}}}{\hat{\tau}_{rel}}-2\hat{\pi}^{{\mu}{\nu}}\tanh\rho\nonumber\\
&-\frac{\tanh\rho}{\hat{T}}\int{\frac{\partial^3\hat{p}}{(2\pi)^3}}\frac{\hat{\Delta}^{{\mu}{\nu}}_{{\alpha}{\beta}}\hat{p}^{\alpha}\hat{p}^{\beta}}{{\sqrt[]{-g}(\hat{p}^{\rho})^2}}(\frac{\hat{p}_{\theta}^2}{\cosh^2\rho}+\frac{\hat{p}_{\phi}^2}{\cosh^2\rho\sin^2\theta})e^{\frac{-\hat{p}^{\rho}}{\hat{T}(\rho)}}\nonumber\\
&+\int{\frac{\partial^3\hat{p}}{(2\pi)^3}}\frac{\hat{\Delta}^{{\mu}{\nu}}_{{\alpha}{\beta}}\hat{p}^{\alpha}\hat{p}^{\beta}}{{\sqrt[]{-g}(\hat{p}^{\rho})^2}}\frac{1}{\sqrt[]{-g}}\frac{1}{\hat{p}^\rho}\frac{\partial\hat{p}}{\partial\rho}\delta{f}
\label{shear_con_integrales}
\end{eqnarray}

El único problema acá es obtener la segunda linea  que es en la que se integra la derivada de $f_{eq}$, donde se ve que lo que esta entre paréntesis es $\hat{p}_{\Omega}^2/\cosh^2\rho=(\hat{p}^{\rho})^2-(\hat{p}^{\eta})^2$.\\

{\Large{\textbf{\textit{ No puedo recuperar la segunda linea de (\ref{shear_con_integrales}) a partir de la (\ref{derivada_shear_desarrollada}). El problema esta en que no puedo recuperar el factor $\tanh\rho$ y me falta un $(\hat{p}^{\eta})^2$ en la integral}}}}\\

Como solo nos interesa la dinámica de la única componente independiente del tensor de esfuerzos viscosos, evaluamos los indices de \ref{shear_con_integrales} en $\mu=\nu=\eta$ quedando:

\begin{eqnarray}
\partial_{\rho}\hat{\pi}^{<{\mu}{\nu}>}=
-\frac{\hat{\pi}^{{\eta}{\eta}}}{\hat{\tau}_{rel}}
+\frac{4}{3}\frac{\hat{\eta}}{\hat{\tau}}\tanh\rho
-\frac{46}{21}\hat{\pi}^{\eta\eta}\tanh\rho\nonumber\\
+\frac{\tanh\rho}{3(2\pi)}\int_{0}^{\infty}\partial\hat{p}^{\rho}\int_{0}^{2\pi}\partial\theta(\hat{p}^{\rho})^3\sin\theta(\frac{25}{21}-\cos^2\theta)(3\cos^2\theta-1)\delta{f}
\label{shear_exacta}
\end{eqnarray}

La ecuación sin el ultimo término nos da la evolución conforme de Israel-Stewart a segundo orden para la componente independiente de $\hat{pi}^{{\mu}{\nu}}$. El ultimo termino es la corrección que viene del tratamiento exacto de la función distribución. Esta corrección puede perderse si se usa un método aproximado para llegar a las ecuaciones.



\subsection{Ecuación a segundo orden para el tensor de esfuerzos viscosos}


En el espacio de $ds_{3}{\otimes}R$ con la métrica $\hat{g}_{{\mu}{\nu}}=diag(-1,\cosh^2\rho,\cosh^2\rho\sin^2\theta,1)$, el ritmo de expansión es se la forma $\hat{\theta}=\hat{\nabla}_{\mu}\hat{u}^{\mu}=\frac{1}{\sqrt[]{-g}}\partial_{\mu}(\sqrt[]{-g}\hat{u}^{\mu})$ y el tensor de esfuerzos de corte es $\sigma_{{\mu}{\nu}}=\hat{\Delta}^{{\mu}{\nu}}_{{\alpha}{\beta}}\hat{\nabla}_{\alpha}\hat{u}_{\beta}$ que se puede ver que es de la forma:

\begin{equation}
\sigma_{{\mu}{\nu}}=\Gamma^{\rho}_{{\mu}{\nu}}-\frac{1}{3}\hat{\Delta}_{{\mu}{\nu}}\hat{\theta}= diag(0,\frac{1}{3}\cosh\rho\sinh\rho,\frac{1}{3}\sin^\rho\cosh\rho\sinh\rho,-\frac{2}{3}\tanh\rho)
\label{esfuerzp_de_corte}
\end{equation}

La ecuación para el tensor de esfuerzos viscosos a segundo orden es:

\begin{equation}
\hat{\tau}_{\pi}\hat{\Delta}^{{\alpha}{\beta}}_{{\mu}{\nu}}\hat{u}^{\lambda}\hat{\nabla}_{\lambda}\hat{\pi}_{{\alpha}{\beta}}+\hat{\pi}_{{\mu}{\nu}}=-2\hat{\eta}\hat{\sigma}_{{\mu}{\nu}}-\frac{4}{3}\hat{\tau}_{\pi}\hat{\pi}_{{\mu}{\nu}}\hat{\theta}-\frac{10}{7}\hat{\tau}_{\pi}\hat{\pi}^{\lambda}_{<\mu}\hat{\sigma}_{{\nu>}{\lambda}}
\label{shear_segundo_orden}
\end{equation}

El termino con $\hat{\pi}^{\lambda}_{<\mu}\hat{\sigma}_{{\nu>}{\lambda}}$ se simplifica desarrollando $\hat{\Delta}^{{\mu}{\nu}}_{{\alpha}{\beta}}$

\begin{equation}
\hat{\pi}^{\lambda}_{<\mu}\hat{\sigma}_{{\nu>}{\lambda}}=\hat{\Delta}^{{\alpha}{\beta}}_{{\mu}{\nu}}\hat{\pi}^{\lambda}_{\alpha}\hat{\sigma}_{{\beta}{\lambda}}=
\frac{1}{2}(\hat{g}^{\mu}_{\alpha}\hat{g}^{\nu}_{\beta}+\hat{g}^{\mu}_{\beta}\hat{g}^{\nu}_{\alpha}-\frac{2}{3}\hat{g}^{{\alpha}{\beta}}\hat{\Delta}_{{\mu}{\nu}})\hat{\pi}^{\lambda}_{\alpha}\hat{\sigma}_{{\beta}{\lambda}}
\end{equation}

Donde en el último paso se pudieron reemplazar proyectores por métricas debido a que tanto $\hat{\pi}^{\lambda}_{\alpha}$ y $\hat{\sigma}_{{\beta}{\lambda}}$ son ortogonales al vector cuadrivelocidad. Finalmente este termino resulta:

\begin{equation}
\hat{\pi}^{\lambda}_{<\mu}\hat{\sigma}_{{\nu>}{\lambda}}=
\frac{1}{2}\hat{\pi}^{\lambda}_{\mu}\hat{\sigma}_{{\nu}{\lambda}}
+\frac{1}{2}\hat{\pi}^{\lambda}_{\nu}\hat{\sigma}_{{\mu}{\lambda}}
+\frac{1}{3}\hat{\Delta}_{{\mu}{\nu}}\hat{\pi}^{{\eta}{\eta}}\tanh\rho
\end{equation}

Donde el término que nos va a interesar mas adelante es $\hat{\pi}^{\lambda}_{<\eta}\hat{\sigma}_{{\eta>}{\lambda}}=-\frac{1}{3}\hat{\pi}^{\eta}_{\eta}\tanh\rho$.


El termino de relajación $\hat{\Delta}^{{\alpha}{\beta}_{{\mu}{\nu}}}\hat{u}^{\lambda}\hat{\nabla}_{\lambda}\hat{\pi}_{{\alpha}{\beta}}$ en (\ref{shear_segundo_orden}) se puede desarrollar (por partes) de la siguiente forma:

\begin{equation}
\hat{\Delta}^{{\mu}{\nu}}_{{\alpha}{\beta}}\hat{u}^{\lambda}\hat{\nabla}_{\lambda}\hat{\pi}_{{\alpha}{\beta}}=
\hat{D}\hat{\pi}_{{\mu}{\nu}}-
\hat{\pi}_{{\alpha}{\beta}}\hat{D}\hat{\Delta}^{{\alpha}{\beta}}_{{\mu}{\nu}}
\end{equation}

El primer termino da

\begin{equation}
\hat{D}\hat{\pi}_{{\mu}{\nu}}=
\hat{u}^{\lambda}\partial_{\lambda}\hat{\pi}_{{\mu}{\nu}}
-\hat{u}^{\lambda}\Gamma^{\alpha}_{{\mu}{\lambda}}\hat{\pi}_{{\alpha}{\nu}}
-\hat{u}^{\lambda}\Gamma^{\alpha}_{{\lambda}{\nu}}\hat{\pi}_{{\mu}{\alpha}}
\label{derivada_covariante_del_shear}
\end{equation}

Debido a que cuando evaluamos en el caso que nos interesa $\mu=\nu=\eta$ los símbolos de Christoffel son nulos este termino no aporta a la dinámica de $\pi_{{\eta}{\eta}}$. Algo similar ocurre para el segundo termino de (\ref{derivada_covariante_del_shear}):

\begin{equation}
\hat{\pi}_{{\alpha}{\beta}}\hat{D}\hat{\Delta}^{{\alpha}{\beta}}_{{\mu}{\nu}}=(\hat{u}_{\nu}\hat{\pi}^{\alpha}_{\mu}+\hat{u}_{\mu}\hat{\pi}^{\alpha}_{\nu})\hat{u}^{\sigma}\Gamma^{\lambda}_{{\sigma}{\alpha}}\hat{u}_{\lambda}
\end{equation}

Finalmente la ecuación  para la dinámica de $\pi_{{\eta}{\eta}}$ queda:

\begin{equation}
\hat{\tau}_{\pi}\partial_{\rho}\hat{\pi}_{{\eta}{\eta}}
+\hat{\pi}_{{\eta}{\eta}}=
(\frac{4}{3}\hat{\eta}-\frac{8}{3}\hat{\tau}_{\pi}\hat{\pi}_{{\eta}{\eta}}
+\frac{10}{21}\hat{\tau}_{\pi}\hat{\pi}_{{\eta}{\eta}})\tanh\rho
\end{equation}

Esta ecuación se puede reescribir utilizando que $\hat{\eta}=\frac{\hat{S}}{5}\hat{\tau}_{rel}\hat{T}=\frac{\hat{\epsilon}+\hat{P}}{5}\hat{\tau}_{rel}=\frac{4}{15}\hat{\epsilon}\hat{\tau}_{rel}$, quedando.

\begin{equation}
\partial_{\rho}\hat{\pi}+\frac{\hat{\pi}}{\hat{\tau}_{rel}}
+\frac{46}{21}\hat{\pi}\tanh\rho=
\frac{16}{45}\hat{\epsilon}\tanh\rho
\label{ecuacion_shear}
\end{equation}

Definiendo el tensor reescaleado $\hat{\bar{\pi}}=\hat{\pi}^{\eta}_{\eta}/(\hat{\epsilon}+\hat{P})=\frac{3}{4}\frac{\hat{\pi}}{\hat{\epsilon}}$ se pueden desacoplar las ecuaciones (\ref{ecuacion_energia}) y (\ref{ecuacion_shear}). Multiplicando la ecuación (\ref{ecuacion_energia}) por $\frac{3}{4}\frac{1}{\hat{\epsilon}}$ y haciendo un poco de álgebra se llega a la siguiente ecuación para la densidad de energía:

\begin{equation}
\partial_{\rho}\ln\hat{\epsilon}=
\frac{4}{3}(\hat{\bar{\pi}}-2)\tanh\rho
\end{equation}

Escribiendo a $\hat{\pi}=\frac{4}{3}\hat{\bar{\pi}}\hat{\epsilon}$ y usando la regla del producto en el primer termino de (\ref{ecuacion_shear}) se obtiene:

\begin{equation}
\frac{4}{3}\epsilon\partial_{\rho}\hat{\bar{\pi}}
+\frac{4}{3}\hat{\bar{\pi}}\partial_{\rho}\epsilon
+\frac{\hat{\pi}}{\hat{\tau}_{rel}}
+\frac{46}{21}\hat{\pi}\tanh\rho=
\frac{16}{45}\hat{\epsilon}\tanh\rho
\end{equation}

Finalmente multiplicando por $\frac{3}{4}\frac{1}{\hat{\epsilon}}$ se llega a una ecuación que solo depende de $\hat{\bar{\pi}}$:

\begin{equation}
\partial_{\rho}\hat{\bar{\pi}}+\frac{\hat{\bar{\pi}}}{\hat{\tau}_{rel}}=
\frac{4}{3}\tanh\rho(\frac{1}{5}+\frac{5}{14}\hat{\bar{\pi}}-\hat{\bar{\pi}}^2)
\end{equation}






%(\ref{tercterm}) conviene desarrollar $\hat{\Delta}^{{\mu}{\nu}}_{{\alpha}{\beta}}$ teniendo en cuenta que $\hat{u}^{\nu}$




%\begin{thebibliography}{10}

%\bibitem{Durrer} R. Durrer, A. Neronov "Cosmological Magnetic Fields: Their Generation, Evolution and Observation" 	arXiv:1303.7121 [astro-ph.CO]Appl. 
%para citar es \cite{}

%\end{thebibliography}



\end{document}